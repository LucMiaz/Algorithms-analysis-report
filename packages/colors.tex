%ajoute des couleurs
\usepackage{xcolor}
\definecolor{bleurusse}{HTML}{5A8BBA}
\definecolor{orangerusse}{HTML}{E84E0F}
\definecolor{orangebrewer}{RGB}{226,115,0}

%change la marge des légendes:
\usepackage{caption, xcolor,tabu}
\colorlet{captionlabel}{bleurusse}

\captionsetup{
  font=footnotesize,
  format=hang,
  singlelinecheck=false,
  labelfont={bf, color=captionlabel}
}

%signes et style de listes et enumerations
\renewcommand{\labelitemi}{$\bullet$}
\renewcommand{\labelitemii}{$\cdot$}
\renewcommand{\labelitemiii}{$\diamond$}
\renewcommand{\labelitemiv}{$\ast$}
%
\renewcommand{\labelenumi}{(\arabic{enumi}.}
\renewcommand{\labelenumii}{\alph{enumii})}
\usepackage[hang,nooneline]{subfigure}

%contols float placements: \FloatBarrier to ensure all floats for a section appear before the next \section 
\usepackage{placeins}

%gestion des éléments graphiques
\usepackage{graphicx}\newcommand{\HRule}{\rule{\linewidth}{0.25pt}}
%-----------------------------------%
%----------listings :---------------%
%------affichage du code-----%
%-----------------------------------%

%importation
\usepackage{listings}
%instantiation des options style général
\lstset{ %
  backgroundcolor=\color{white},   % couleur de fond
  basicstyle=\footnotesize\ttfamily,,     % taille et police
  belowcaptionskip=3mm,            % modifie la distance entre la légende et le listing
  breakatwhitespace=true,          % coupure uniquement aux espaces
  breaklines=true,                % retour à la ligne automatique
  captionpos=t,                    % position de la légende
  commentstyle=\color{vert},       % couleur des commentaires
  extendedchars=true,              % permets l'affichage de caractères non ASCII
  framerule=0.25pt,				  % épaisseur du cadre
  framesep=1pt,                    % définit la distance entre le cadre et le texte
  frameshape={NNN}{n}{n}{YYY},     % ajoute un cadre autour du code (single, trblTRBL) {<top>}{<left>}{<right>}{<bottom>}, avec Y, N ou R (pour arrondi) pour top et bottom, 3 lettres par étage, lu depuis le centre; y ou n pour left et right, une lettre par étage. Cf listing package doc p. 34.
  framexbottommargin=0pt,          % modifie la marge du bas du contour
  framexleftmargin=0pt,            % modifie la marge de gauche du contour
  framexrightmargin=0pt,           % modifie la marge de droite du contour
  keepspaces=true,                 % garde les espaces dans le code (indentation)
  keywordstyle=\bfseries\color{rouge},       % style des mots clefs scilab
  language=Python,                 % langage
  linewidth=\linewidth,
  numbers=left,                    % emplacement des numéros de ligne
  numbersep=5pt,                   % marge entre les numéros de ligne et le code
  numberstyle=\tt\tiny\color{gris},  % style pour les numéros de ligne
  resetmargins=false,			  % remet ou non les marges à 0 si dans une liste
  rulecolor=\color{bleu},          % couleur du cadre
  rulesep=1pt,                     % définit la distance entre les lignes du cadre
  %rulesepcolor=\color{bleu!50},   % couleur de remplissage entre les doubles lignes du cadre
  showspaces=false,                % ne pas souligner tous les espaces
  showstringspaces=false,          % ne pas souligner les espaces
  showtabs=false,                  % Ne pas indiquer les tabulations dans un string
  stepnumber=1,                    % degré de numérotation des lignes (1 = numérote toutes les lignes)
  stringstyle=\color{orange!50},   % style des éléments 'string'
  tabsize=4,                       % taille des tabulations en espace
  xleftmargin=1pt,                 % modifie la marge de gauche
  xrightmargin=1pt,                % modifie la marge de droite
}
\lstdefinestyle{myR}{ %
  backgroundcolor=\color{white},   % couleur de fond
  basicstyle=\footnotesize\ttfamily,,     % taille et police
  belowcaptionskip=3mm,            % modifie la distance entre la légende et le listing
  breakatwhitespace=true,          % coupure uniquement aux espaces
  breaklines=true,                % retour à la ligne automatique
  captionpos=t,                    % position de la légende
  commentstyle=\color{vert},       % couleur des commentaires
  extendedchars=true,              % permets l'affichage de caractères non ASCII
  framerule=0.25pt,				  % épaisseur du cadre
  framesep=1pt,                    % définit la distance entre le cadre et le texte
  frameshape={NNN}{n}{n}{YYY},     % ajoute un cadre autour du code (single, trblTRBL) {<top>}{<left>}{<right>}{<bottom>}, avec Y, N ou R (pour arrondi) pour top et bottom, 3 lettres par étage, lu depuis le centre; y ou n pour left et right, une lettre par étage. Cf listing package doc p. 34.
  framexbottommargin=0pt,          % modifie la marge du bas du contour
  framexleftmargin=0pt,            % modifie la marge de gauche du contour
  framexrightmargin=0pt,           % modifie la marge de droite du contour
  keepspaces=true,                 % garde les espaces dans le code (indentation)
  keywordstyle=\bfseries\color{bleu},       % style des mots clefs R
  language=R,                 % langage
  linewidth=\linewidth,
  numbers=left,                    % emplacement des numéros de ligne
  numbersep=5pt,                   % marge entre les numéros de ligne et le code
  numberstyle=\tt\tiny\color{gris},  % style pour les numéros de ligne
  resetmargins=false,			  % remet ou non les marges à 0 si dans une liste
  rulecolor=\color{bleu},          % couleur du cadre
  rulesep=1pt,                     % définit la distance entre les lignes du cadre
  %rulesepcolor=\color{bleu!50},   % couleur de remplissage entre les doubles lignes du cadre
  showspaces=false,                % ne pas souligner tous les espaces
  showstringspaces=false,          % ne pas souligner les espaces
  showtabs=false,                  % Ne pas indiquer les tabulations dans un string
  stepnumber=1,                    % degré de numérotation des lignes (1 = numérote toutes les lignes)
  stringstyle=\color{orange!50},   % style des éléments 'string'
  tabsize=4,                       % taille des tabulations en espace
  xleftmargin=1pt,                 % modifie la marge de gauche
  xrightmargin=1pt,                % modifie la marge de droite
}
%change la couleur des mots clefs de scilab
\newcommand{\CodeBrackets}[1]{{\color{bleu}{#1}}}    % définit une commande pour changer la couleur des parenthèses et crochets
\newcommand{\CodeDigits}[1]{{\color{orange}{#1}}}    % définit une commande pour colorer les chiffres
\newcommand{\CodeSymb}[1]{{\color{orange!50}{#1}}}   % définit une commande pour colorer les opérateurs

\lstset{literate=%
% digits
	{0}{{\CodeDigits{0}}}1
    {1}{{\CodeDigits{1}}}1
    {2}{{\CodeDigits{2}}}1
    {3}{{\CodeDigits{3}}}1
    {4}{{\CodeDigits{4}}}1
    {5}{{\CodeDigits{5}}}1
    {6}{{\CodeDigits{6}}}1
    {7}{{\CodeDigits{7}}}1
    {8}{{\CodeDigits{8}}}1
    {9}{{\CodeDigits{9}}}1
% parenthèses
	{)}{{\CodeBrackets{)}}}1
    {(}{{\CodeBrackets{(}}}1
    {[}{{\CodeBrackets{[}}}1
    {\{}{{\CodeBrackets{\{}}}1
    {]}{{\CodeBrackets{]}}}1
    {\}}{{\CodeBrackets{\}}}}1
% ponctuation
    {:}{{\CodeSymb{: }}}1
        }
% le patch suivant corrige un bug concernant la parenthèse fermante et l'option "breaklines"
% source cf. http://tex.stackexchange.com/q/69472   
\usepackage{etoolbox}
\makeatletter
\patchcmd{\lsthk@SelectCharTable}{%
  \lst@ifbreaklines\lst@Def{`)}{\lst@breakProcessOther)}\fi}{}{}{}
\makeatother


% change la marge et la couleur des légendes:
\usepackage{float}
\usepackage{caption,tabu}
%\captionsetup[table]{name=Tableau}% modifie le titre donné aux légendes des 'table'
\floatstyle{boxed}\restylefloat{figure}%ajoute un cadre aux figures
\captionsetup{%modifie l'affichage des légendes des Floats, notamment en colorant le titre du float en bleu.
  font=footnotesize,
  format=hang,
  singlelinecheck=false,
  labelfont={bf,sf, color=rouge}
}

%change la numérotation des sous-figures
\renewcommand{\thesubfigure}{\roman{subfigure}.}
\makeatletter
\renewcommand\p@subfigure{\thefigure\,}%ajoute un espace entre le numéro de la figure et celui de la sous-figure
\makeatother
%\captionsetup[subfigure]{labelfont={bf,sf,color=orange},labelformat=simple, labelsep=colon, textfont=normalfont, font=footnotesize}

% change les légendes des listings
\DeclareCaptionFont{white}{\color{white}}
\DeclareCaptionFormat{listing}{{\colorbox{vert}{\parbox{\linewidth-6pt}{\hspace{5pt}#1#2#3}}}}
\captionsetup[lstlisting]{format=listing,labelfont=white,textfont=white, singlelinecheck=false, indention=0pt, margin=0pt, font={bf,sf,small}}

% changement du préfixe du float lstlisting (défaut=listing)
\renewcommand{\lstlistingname}{Code} % préfixe dans les légendes de listing
\renewcommand{\lstlistlistingname}{Codes} % préfixe dans les listes de listing

% création d'une commande formattant un élément donné comme une variable dans le code scilab
\newcommand{\code}[1]{{\tt\bf\color{brown}{#1}}}

%gestion des commentaires
\usepackage{comment} %use \begin{comment} commentaire \end{comment}
\input{bibliographie.tex}
\bibliography{SBBbib.bib}